%-------------------------------------------------------------------------------
\section{Installation procedure}\label{sec:installation}
%-------------------------------------------------------------------------------
%-------------------------------------------------------------------------------
\subsection{Requirements}\label{sec:requirements}
%-------------------------------------------------------------------------------
The installation of the basic SKYCORR binary package requires:
\begin{itemize}
  \item
    C99 compatible compiler (e.g. gcc or clang)
  \item
    glibc 2.11 or newer on Linux or OS X 10.7 or newer
  \item
    common unix utilities (bash, tar, sed, grep, \ldots{})
\end{itemize}

The optional \ac{GUI} to SKYCORR requires:

\begin{itemize}
  \item
    Python v2.6 or v2.7 (but not Python v3.x)
  \item
    wxPython v2.8 or newer
  \item
    Python matplotlib v1.0 or newer
  \item
    PyFITS v2.4 or newer
\end{itemize}

The command line client also has optional display features which
require:

\begin{itemize}
  \item gnuplot v4.2 patchlevel 3 or newer
\end{itemize}

%-------------------------------------------------------------------------------
\subsection{Binary installation}
\label{sec:installscript}
%-------------------------------------------------------------------------------

First the downloaded installer needs to be made executable. To do this change
into the directory the installer was downloaded to and run following command
(replacing {\tt skycorr\_installer.run} with the actual downloaded filename):

\begin{verbatim}
chmod u+x ./skycorr_installer.run
\end{verbatim}

Now the installer can be executed from the same folder with:

\begin{verbatim}
./skycorr_installer.run
\end{verbatim}

It will ask for an installation directory where it will extract its
contents to. It is recommended to choose an empty directory to avoid
overwriting existing files.

After the installer has successfully finished, the {\tt skycorr} executables
are installed into the \texttt{bin} subdirectory of the chosen installation
folder. They can be executed by specifying the full or relative path.
Also installed are a set of example parameter files for several
instruments in the \texttt{examples/config} directory. To run a SINFONI example
type:

\begin{verbatim}
<INST_DIR>/bin/skycorr <INST_DIR>/examples/config/sctest_sinfo_H.par
\end{verbatim}

For more details see Section~\ref{sec:running}.

The following directory structure is created by the installation routine:

\begin{verbatim}
   <INST_DIR>
        |
        |-- bin
        |-- config
        |-- doc
        |-- examples
        |       |-- config
        |       |-- data
        |       |-- sysdata -> <INST_DIR>/sysdata/
        |-- output
        |-- sysdata

\end{verbatim}
In detail:
\begin{itemize}
  \item {\tt bin/}: location of binary files
  \item {\tt config/}: directory containing template configuration files
  \item {\tt doc/}: documentation
  \item {\tt examples/}: directory with examples
  \item {\tt output/}: default output directory
  \item {\tt sysdata/}: directory containing data required for SKYCORR
\end{itemize}


%-------------------------------------------------------------------------------
\subsection{\ac{GUI} dependencies}
\label{sec:guidependencies}
%-------------------------------------------------------------------------------

The \ac{GUI} requires some additional dependencies to be installed on the system. To
check if the python installation is able to run the \ac{GUI}, following
commands can be run:

\begin{verbatim}
python -c 'import wx'

python -c 'import matplotlib; import matplotlib.backends.backend_wxagg'

python -c 'import pyfits'
\end{verbatim}

If these commands fail please see following site for instructions on how
to install these packages:

\texttt{http://www.eso.org/pipelines/reflex\_workflows/}

%-------------------------------------------------------------------------------
\subsection{Package contents}
\label{sec:pkgcontents}
%-------------------------------------------------------------------------------

The installation package is a self extracting tarball containing the
SKYCORR source code and pre-built versions of its third party dependencies:

\begin{itemize}
  \item
    Common Pipeline Library v6.4.2 and its dependencies cfitsio v3.350,
    wcslib v4.16 and fftw3 v3.3.3 \cite{CPL}
\end{itemize}


%-------------------------------------------------------------------------------
\subsection{Source Installation}
\label{sec:sourceinstall}
%-------------------------------------------------------------------------------

Advanced users may want to install everything from source, the
basic instructions for this are outlined in this section.

\subsubsection{CPL compilation}

The CPL sources can be obtained from \cite{CPL}

CPL only requires cfitsio in order to run SKYCORR. It can be installed as
follows:

\begin{verbatim}
./configure --prefix=/install-location

make

make shared

make install
\end{verbatim}

Then CPL can be install with:

\begin{verbatim}
./configure --prefix=/install-location --with-cfitsio=/install-location

make

make install
\end{verbatim}

See the respective packages documentation for details on the
installation procedure.

\subsubsection{SKYCORR compilation}

After all dependencies have been installed SKYCORR can be compiled from source
into the same location.

This is the only step required if one wants to update SKYCORR from
source after previously installing the third party dependencies with the
binary installer.

\begin{verbatim}
./configure --prefix=/install-location --with-cpl=/install-location

make

make install
\end{verbatim}

In order to use SKYCORR from this location the environment variable
{\tt LD\_LIBRARY\_PATH} \\(or {\tt DYLD\_LIBRARY\_PATH} on Mac OS) need to be
set. With the bash shell this is done with following command:

\begin{verbatim}
export LD_LIBRARY_PATH=/install-location/lib
\end{verbatim}

Now {\tt skycorr} is ready to be used from {\tt /install-location/bin}.
